\documentclass{article}

\usepackage{geometry}
\usepackage{fancyhdr}
\usepackage{amsmath, amssymb}
\usepackage{graphicx}
\usepackage{color}
\definecolor{bg}{rgb}{0.89, 0.95, 0.71} % background e3f3b4
\definecolor{ac}{rgb}{0.50, 0.18, 0.41} % accent c491b6
\definecolor{rc}{rgb}{0.77, 0.57, 0.71} % rule 7f2f69
\usepackage{listings}
%catalan packages
\usepackage[catalan]{babel}
\usepackage[T1]{fontenc}
\usepackage[utf8]{inputenc}
%
\lstset{
       basicstyle=\small\ttfamily,
       xleftmargin=1em,
       }
\lstdefinestyle{latex}{language=TeX,
                       backgroundcolor=\color{bg},
                       basicstyle=\small\ttfamily,
                       frame=leftline,
                       xleftmargin=1.4em, 
                       framexleftmargin=.8em}
\lstdefinestyle{cmdline}{
                         }
\usepackage{url}
\usepackage[pdftex, 
            bookmarks, 
            colorlinks=true,
            linkcolor=black, 
            plainpages = false,
            pdfpagemode = UseNone,
            pdfstartview = FitH, 
            citecolor = ac, urlcolor = ac, filecolor = ac]{hyperref}

\usepackage[para]{footmisc}
% Change horiz room between fn mark and fn hskip from .5em 
% Suggested to RF making this settable
\makeatletter
\long\def\@makefntext#1{\leavevmode
\@makefnmark\nobreak
\hskip.05em\relax#1%
}
\makeatother
%\newcommand{\texdoc}[1]{\/\footnote{\protect\texttt{#1}}}
\newcommand{\citelink}[3]{\href{#1}{#2}~\cite{#3}}
\newcommand{\bibliolink}[2]{\href{#1}{\nolinkurl{#1}}, \href{#1}{#2}}

\setlength{\parskip}{0.75ex}
\makeatletter % from emma pease at csli.stanford.edu
% \@startsection {NAME}{LEVEL}{INDENT}{BEFORESKIP}{AFTERSKIP}{STYLE} 
%            optional * [ALTHEADING]{HEADING}
%    Generic command to start a section.  
%    NAME       : e.g., 'subsection'
%    LEVEL      : a number, denoting depth of section -- e.g., chapter=1,
%                 section = 2, etc.  A section number will be printed if
%                 and only if LEVEL < or = the value of the secnumdepth
%                 counter.
%    INDENT     : Indentation of heading from left margin
%    BEFORESKIP : Absolute value = skip to leave above the heading.  
%                 If negative, then paragraph indent of text following 
%                 heading is suppressed.
%    AFTERSKIP  : if positive, then skip to leave below heading,
%                       else - skip to leave to right of run-in heading.
%    STYLE      : commands to set style
%  If '*' missing, then increments the counter.  If it is present, then
%  there should be no [ALTHEADING] argument.  A sectioning command
%  is normally defined to \@startsection + its first six arguments.
%
%colors in sections and subsections
%\def\section{\@startsection {section}{1}{\z@}{2.5ex plus .6ex minus 
%    .2ex}{1.0ex plus .15ex}{\hspace*{-3em}\Large\bf\color{ac}}}
%\def\subsection{\@startsection{subsection}{2}{\z@}{1.5ex plus .3ex minus 
%   .1ex}{.2ex plus .1ex}{\hspace*{-3em}\bf\large\color{ac}}}
\makeatother
\setcounter{secnumdepth}{0}

\newlength{\rulelength}
\setlength{\rulelength}{\linewidth}
\addtolength{\rulelength}{9.5em}
\title{Matemàtiques 1r batxillerat científic}
\author{Mireia Dosil Bonmatí}
\author{Institut de Vilafant. Curs 15}

\pagestyle{plain}
\begin{document}
\thispagestyle{empty}
% \maketitle
\setlength{\unitlength}{1in}

\makeatletter
\vspace*{3ex}
\par\noindent{\hspace*{-1.5em}\LARGE\bf \@title}
\vspace*{-1.2ex}
\par\noindent{{\hspace*{-1.5em}\rule{\rulelength}{1.05pt}}}
\vspace*{-.5ex}
\par\noindent{\hspace*{-1.5em}\large \@author}
\vspace*{5ex}
\makeatother

\section{Funcionament del curs}

Utilitzarem el moodle al llarg del curs per a entrega de tasques i presentació de materials. Com que no hi ha llibre, els apunts els podreu trobar en aquest enllaç: \emph{http://mdosil.cat/mates1batcientific/}

El curs està dividit en 3 blocs diferenciats que intentarem fer coincidir amb els trimestres. 




\begin{center}
	\begin{tabular}{ | l |}
		\hline
		\textbf{Bloc 1: Nombres. Polinomis}   \\ 
		\hline
		1. Nombres reals \\
		2. Trigonometria\\
		3. Nombres complexos\\
		4. Polinomis\\
		\hline
		\hline
		\textbf{Bloc 2: Geometria}   \\ 
		\hline
		5. Vectors en el pla \\
		6. Rectes en el pla\\
		7. Llocs geomètrics\\
		\hline
		\hline
		\textbf{Bloc 3: Funcions}   \\ 
		\hline
		8. Funcions \\
		9. Successions\\
		10. Límits i continuïtat de funcions\\
		11. Funció exponencial i logarítmica\\
		12. Funcions trigonomètriques\\
		\hline
		
	\end{tabular}
\end{center}


\section{Normativa d'exàmens}

\begin{itemize}
\item Hi haurà com a mínim dues proves escrites de cada bloc. El 90\% de la nota del trimestre serà la mitjana aritmètica de les proves escrites. L'altre 10\% restant el formaran l'actitud i les activitats complementàries.
\item Els exàmens seran en bolígraf, no es podran fer preguntes i s'utilitzarà el propi material (calculadora, regle, etc)
\item Si un alumne/a falta les hores de classe abans de l'examen prerdrà el dret d'examinar-se si no ho justifica de manera oficial
\item Si un alumne/a falta el dia de l'examen ha de portar un justificant oficial per tenir dret a fer l'examen un altre dia.
\end{itemize}

\section{Puntuació}

\begin{itemize}
	\item Nota trimestral: Controls (mitjana aritmètica, 90\%) + Actitud (10\%)
	\item Nota Curs: Mitjana aritmètica dels tres trimestres (han d'estar aprovats tots tres o només un trimestre suspès amb una nota superior a 3)
	\item Els deures d'estiu són un 10\% de la nota del primer trimestre
\end{itemize}

\section{Recuperacions}

\begin{itemize}
	\item Es recupera per trimestres (els 15 dies següents d'acabar cada trimestre)
	\item Al juny hi haurà una recuperació final de tot el curs. En cas de no aprovar la recuperació, l'alumne/a té dret a la prova extraordinària de setembre
	\item La nota de recuperació serà sempre un \bfseries{5}
	\normalfont
	\item Al juny hi haurà un examen opcional per a pujar nota
\end{itemize}






\end{document}
