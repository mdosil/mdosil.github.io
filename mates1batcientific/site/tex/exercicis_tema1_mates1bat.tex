\documentclass{article}
\usepackage{fancyhdr}
\usepackage[catalan]{babel}
\usepackage[T1]{fontenc}
\usepackage[utf8]{inputenc}



 
\pagestyle{fancy}
\fancyhf{}

\rhead{Exercicis}
\lhead{Tema 1. Nombres reals.}
\lfoot{Institut de Vilafant. Matemàtiques 1 batxillerat científic}
\rfoot{\thepage}
 
\begin{document}
	\author{Mireia Dosil}
	\date{text}


\begin{enumerate}
 
\item Considera els nombres següents: $1$, $-5$, $\sqrt{3}$, $\pi$, $4'15$, $2'\stackrel\frown{3}$, $\frac{4}{5}$.
\begin{enumerate}
	\item Classifica'ls
	\item Representa'ls sobre la recta real
	\item Ordena'ls de menor a major
\end{enumerate}

\item Dóna dos nombres racionals que es trobin entre $\sqrt{3}$ i $1'73$.
\item Com ho faries per representar el nombre $\sqrt{5}$ sobre la recta real?



\item Representa sobre la recta real els intervals següents. Expressa'ls després en forma de desigualtat i també d'interval.

\begin{enumerate}
\item els nombres més grans o iguals que cinc
\item els nombres que es troben entre $-\frac{5}{6}$ i 0
\item els nombres més petits que 4 i més grans o iguals que 3
\item els nombres més petits que $\sqrt{5}$
\end{enumerate}

\item El nombre 5 pertany a l'interval $\lbrack 5, + \infty) $ ? Per què?
\item Representa i simplifica els intervals següents:

\begin{enumerate}
\item $\lbrack -3, 5 \rbrack \cap (3, 8)$
\item $(- \infty, 3) \cap \lbrack -5, 0 \rbrack$
\item $(- \infty, 5) \cap \lbrack 0, + \infty)$
\end{enumerate}


\item Resol les inequacions següents i dóna el resultat de manera gràfica i en forma d'interval:

\begin{enumerate}
\item $4x-3(x-6) \ge 2x+5$
\item $\mid x-5 \mid \ge 2$ \tiny{ (Entenem $\mid a - b\mid$ com la distància entre $a$ i $b$) }
\normalsize
\item $10 - \frac{1}{2} (x-3) \le -5 -x$



\end{enumerate}

\end{enumerate}
 
\end{document}